%%% Fonts & Symbols %%%
%LaTeX
\usepackage[utf8]{inputenc} 													% Special characters; LaTeX
\usepackage[T1]{fontenc} 														% Special characters; LaTeX
\fontfamily{lmodern}															% Font; default LaTeX font; LaTeX
\fontfamily{libertine}
\usepackage[letterspace=0]{microtype} 							 				% Shrink spacing between letters; doesn't work with XeLaTeX

%Both LaTeX and XeLaTeX
\usepackage{soul}							 										% Highlighting text (did NOT test with XeLaTex)
\usepackage[portuguese, english]{babel} 										% Language package for special characters
\usepackage[space, extendedchars, multidot]{grffile} 							% Special characters
\usepackage[official]{eurosym} 													% Euro symbol
\usepackage{xcolor}%\usepackage[usenames,dvipsnames,svgnames,table]{xcolor} 						% Font colours; simply use e.g. {\color{red} some text} to colour font; cannot be used with Beamer

%%%% Format Paragraph %%% 
\usepackage{ragged2e} 															% Enables justifying text using {\justify text}
\usepackage{setspace} 															% Setspace respects body, footnotes and captions spacing
\usepackage{multicol} 															% Multicolumn in text
\usepackage[flushmargin,bottom]{footmisc}									% Enables formatting footnotes
%\usepackage{dblfnote} 															% Double column in footnotes
\usepackage[all]{nowidow}														% Prevent isolated words at the end of the page
\setlength{\columnsep}{0.5cm} 													% Set the gap between the columns
\interfootnotelinepenalty=10000 												% Prevents breaking footnotes across pages
\setlength{\parskip}{0.5em} 													% Paragraph Spacing
\setlength{\parindent}{0.25cm} 													% Paragraph Indenting;
%!!USES setspace PACKAGE
%\singlespacing 																% Line Single Spacing
%\onehalfspacing 																% Line 1.5 Spacing
%\doublespacing 																% Line 2.0 Spacing
\setstretch{1} 																	% Line 1.25 Spacing
%\begin{doublespace}\end{doublespace} 											% Specific doublespace environment
%\begin{spacing}{2.5}\end{spacing} 												% Specific customized line spacing environment
%{\addfontfeatures{LetterSpace-6} write text here} 								% Shrink spacing between letters for a spefic part of the document; addfontfeatures serves to add all types of font features; used with fontspec
\setlength\footnotemargin{10pt} 												% Specific footnote margin
\raggedbottom																	% Prevents large blanks in the middle of the page
\usepackage{chngcntr}															% Define counters (equations, footnotes, theorems, etc.)
%\counterwithout{footnote}{chapter}									% Prevent footnotes to restart from 1 at each chapter
%\counterwithout{figure}{chapter}										% Prevent figures to restart from 1 at each chapter
%\counterwithout{table}{chapter}											% Prevent tables to restart from 1 at each chapter
%\makeatletter
%        \@addtoreset{footnote}{part}    							% Reset footnote every part
%        \@addtoreset{figure}{part}    							% Reset footnote every part
%        \@addtoreset{table}{part}    							% Reset footnote every part
%\makeatother
%\renewcommand\thefigure{\arabic{part}.\arabic{figure}}				% Inserts the part number in the figure numbering
%\renewcommand\thetable{\arabic{part}.\arabic{table}}				% Inserts the part number in the table numbering
\usepackage{courier} 															% Code-like font
%\usepackage{tocloft}															% Format TOC
\usepackage{pdfpages}														% Include pdf files keeping the header
\usepackage{lastpage}														% Set \pageref{LastPage} for last page counter

%%%% Math %%%
%\usepackage{amsmath}															% Basic math characters; included in mathtools
\usepackage{amsfonts}															% Fonts from the American Mathematical Society
\usepackage{amsthm}												  				% Theorem package
\usepackage{mathtools}															% Basic math characters
\usepackage{amsbsy} 															% Bold symbols with \pmb
\usepackage{amssymb}															% Basic math symbols
\usepackage{bbm}																% Blackboard-style math characters
\usepackage{units}																% Nice diagonal fractions
%\usepackage{bm}																% Bold sym­bols in maths mode; Conflicts with XeLaTeX
\usepackage{bbm}																% Indicator function
\usepackage{empheq} 															% Emphasize equations
\usepackage{centernot} 															% "Not" can be applied to any symbol; does not work with XeLaTeX
\usepackage{tensor}			 													% Enables ordering of superscripts and subscripts
%\usepackage{mathspec}															% Specify arbitrary fonts for mathematics; Use only with XeLaTeX; Conflicts with amsmath
\usepackage{txfontsb}		 													% Enables usage of varmathbb

%%%% Math: Format Space around enviroments %%%
%\makeatletter
%\g@addto@macro\normalsize{%
%  \setlength\abovedisplayskip{10pt}
%  \setlength\belowdisplayskip{10pt}
%  \setlength\abovedisplayshortskip{10pt}
%  \setlength\belowdisplayshortskip{10pt}
%}
%\makeatother

%%% Math Commands %%% 
\newcommand*\widefbox[1]{\fbox{\hspace{2em}#1\hspace{2em}}}						% Emphasize equations use comand \widefbox
\newcommand{\C}{{\varmathbb{C}}}										% Defines command \C to denote Complex
\newcommand{\R}{{\mathbb{R}}} 													% Defines command \R to denote Reals
\newcommand{\Q}{{\varmathbb{Q}}} 										% Defines command \Q to denote Rationals
\newcommand{\Z}{{\varmathbb{Z}}}										% Defines command \Z to denote Integers
\newcommand{\N}{{\mathbb{N}}} 													% Defines command \N to denote Naturals
\newcommand{\pd}{\partial} 														% Defines command \pd to denote partial derivatives
\newcommand{\bimpl}{{\,\Leftrightarrow\,}} 										% Defines command \bimpl to denote long left-right arrow (bidirectional implication)
\newcommand{\nbimpl}{{\,\nLeftrightarrow\,}} 										% Defines command \nbimpl to denote crossed long left-right arrow  (not bidirectional implication)
\newcommand{\impl}{{\,\Rightarrow\,}} 										% Defines command \impl to denote long right arrow (implies)
\newcommand{\nimpl}{{\,\nRightarrow\,}} 										% Defines command \nimpl to denote crossed long right arrow (does not imply)
\newcommand{\limpl}{{\,\Leftarrow\,}} 										% Defines command \limpl to denote long left arrow (left-directional implication)
\newcommand{\rar}{{\,\rightarrow\,}} 										% Defines command \rar to denote right arrow 
\newcommand{\nrar}{{\,\nrightarrow\,}} 										% Defines command \nrar to denote crossed right arrow
\newcommand{\lar}{{\,\leftarrow\,}} 										% Defines command \lar to denote left arrow 
\newcommand{\nlar}{{\,\nleftarrow\,}} 										% Defines command \nlar to denote crossed left arrow
\newcommand{\uar}{{\,\uparrow\,}} 										% Defines command \uar to denote up arrow 
\newcommand{\nuar}{{\,\nuparrow\,}} 										% Defines command \nuar to denote crossed up arrow
\newcommand{\dar}{{\,\downarrow\,}} 										% Defines command \dar to denote down arrow 
\newcommand{\ndar}{{\,\ndownarrow\,}} 										% Defines command \ndar to denote crossed down arrow
\newcommand{\lagr}{\mathcal{L}} 												% Defines command \lagr to denote the Lagrangian L
\newcommand{\hami}{\mathcal{H}} 												% Defines command \hami to denote the Hamiltonian H
\newcommand{\inc}{\Delta\,} 														% Defines command \inc to denote Delta for increment/variation symbol
\newcommand{\ce}{\varepsilon} 														% Defines command \ce to denote curly epsilon
\newcommand{\cphi}{\varphi} 														% Defines command \cphi to denote curly phi
\newcommand{\bcs}{\backslash} 														% Defines command \bcs to denote backslash in math environment
\newcommand{\ints}{\,\mathrm{int}\,}													% Defines command \ints to denote int [interior points of a set] in math environment
\newcommand{\thf}{\,\therefore\,}													% Defines command \thf to denote \therefore in math environment
\newcommand{\bec}{\,\because\,}													% Defines command \bec to denote \because in math environment
\newcommand{\E}{\mathbb{E}}													% Defines command \E to denote E() in math environment
\newcommand{\Var}{\mathbb{V}}													% Defines command \Var to denote Var() in math environment
\newcommand{\Covar}{\text{Cov}}													% Defines command \Covar to denote Covar() in math environment
\newcommand{\supp}{\text{supp}}													% Defines command \supp to denote Var() in math environment
\newcommand{\pr}{\mathbb{P}}													% Defines command \var to denote P() in math environment
\newcommand{\convp}{\overset{p}{\to}}								% Defines command \convp to denote convergence in probability in math environment
\newcommand{\convd}{\overset{d}{\to}}								% Defines command \convp to denote convergence in distribution in math environment
\newcommand{\indeprv}{\protect\mathpalette{\protect\independenT}{\perp}}								% Defines command \indeprv to denote independent random variables in math environment
\def\independenT#1#2{\mathrel{\rlap{$#1#2$}\mkern2mu{#1#2}}}
\newcommand{\oset}{\varnothing} 														% Defines command \oset to denote empty set
\newcommand{\lrp}[1]{\left({#1}\right)}													% Defines command \lrp to denote left-right parenthesis
\newcommand{\lrb}[1]{\left\{{#1}\right\}}													% Defines command \lrb to denote left-right brackets
\newcommand{\lrsb}[1]{\left[{#1}\right]}													% Defines command \lrb to denote left-right brackets
\newcommand{\indic}{\mathbbm{1}}													% Defines command \indic to denote indicator function
\newcommand{\eqnumber}{\refstepcounter{equation}\tag{\theequation}} % Inserts number in equation within nonnumber environment
\newcommand{\dis}{\displaystyle}										 % Renders inline equations as full display

%%% Math Theorems, Propositions and etc. %%% 
%\newtheorem{<name>}{<heading>}[<counter>] 
		%will create an environment <name> for a theorem-like structure; the counter for this structure will be subordinated to <counter>. On the other hand, using
%\newtheorem{<name>}[<counter>]{<heading>}
		%will create an environment <name> for a theorem-like structure; the counter for this structure will share the previously defined <counter> counter.
\newtheorem{thm}{Theorem} 															% Defines the theorem environment
\newtheorem{prop}{Proposition} 											% Defines the proposition environment
\newtheorem{corol}{Corollary} 											% Defines the corollary environment
\newtheorem{claim}{Claim} 														% Defines the corollary environment
\newtheorem{mylemma}{Lemma} 													% Defines the lemma environment
\newtheorem{remark}{Remark} 												% Defines the remark environment
\newtheorem{dfn}{Definition}													% Defines the dfn environment
\newtheorem{conjecture}{Conjecture}													% Defines the dfn environment
\newtheorem{myfact}{Fact}													% Defines the dfn environment
\newtheorem*{myfact*}{Fact}													% Defines the dfn environment
\setbeamertemplate{theorems}[numbered]
%\renewcommand\theprop{\arabic{part}.\arabic{prop}} 						% Inserts the part number in the proposition numbering
%\renewcommand\thetheorem{\arabic{part}.\arabic{theorem}}		% Inserts the part number in the theorem numbering
%\renewcommand\thelemma{\arabic{part}.\arabic{lemma}}			% Inserts the part number in the lemma numbering
%\renewcommand\theremark{\arabic{part}.\arabic{remark}}				% Inserts the part number in the remark numbering

%%% Graphics and Tables %%%
\usepackage{graphicx}															% Support for graphics
\usepackage{tabularx}															% Support for tables
\usepackage{ltablex}															% Long tables X
\usepackage{pdflscape}															% Landscape page contents
\usepackage{longtable} 															% Long tables
\usepackage{tabu} 																% Long tables with longtabu; define paragraphs; flexible tables
\usepackage{hhline} 															% Double hline in tables
\usepackage{booktabs} 															% Format tables; \toprule, \midrule, \bottomrule; informative and practical readme
\usepackage{multirow} 															% Multirow in tables
\usepackage[textfont=bf,hypcap]{caption}		 								% Format captions \label{sec:section} section label~\ref{sec:section}
%\captionsetup[table]{labelformat=simple,labelsep=period}						% Format table captions
%\usepackage[labelformat=empty]{caption} 										% Format captions 
\usepackage{ragged2e}															% Align with \flushleft, etc
\usepackage{array}																% Matricial support
\usepackage{epstopdf}															% Include .eps graphics; does not work well with XeLaTeX
\usepackage{fancyref}															% Labeling details
%\usepackage{tikz-qtree} 														% Draw game theory extensive form graphs/trees
%\usepackage{stata}																% Import tables using sjlog, texdoc
\usepackage{fancyref}															% Details labeling
\usepackage{float}																% Prevent floats from floating
%\usepackage[capposition=top]{floatrow}					% Note for figures using \floatfoot{content...}
\newcommand\fignote[1]{\captionsetup{textfont=normal,font=small}\caption*{#1}}
\usepackage{pgfplots} \pgfplotsset{compat=1.14} \usetikzlibrary{calc}
\newcommand\payoff[1]{ $\begin{pmatrix} #1 \end{pmatrix}$} % Command for payoffs in game trees
\usepackage{dcolumn}	
\usepackage{subcaption}

%%% Lists %%% 
%\usepackage[ampersand]{easylist}												% List package
%\usepackage{enumitem}															% Great list package
									% \begin{enumerate}[label=\roman*] or [label=\alph*] [label=\arabic*] or \Roman* or \Alph* or label=-, \bullet, or any other sign
									% To use embeded lists, use \begin{enumerate}[label=\arabic*.] for the first and \begin{enumerate}[label=\theenumi.(\roman*).] to get 1.i.
									% Anything can be used as a label, namely \bullet, \rhd, \circ, \Box, \rightarrow, \triangleright, \RIGHTarrow - see http://www.rpi.edu/dept/arc/training/latex/LaTeX_symbols.pdf for more symbols
									% It suffices to use \item to get the list
									% [...,noitemsep] kills the space between paragraphs within the list; nosep kills all vertical spacing
									% Resume keeps going at previous list
%\newlist{todolist}{itemize}{2}
%\setlist[todolist]{label=$\square$} % To-do lists
\usepackage{pifont}
\newcommand{\cmark}{\ding{51}}%
\newcommand{\xmark}{\ding{55}}%
\newcommand{\done}{\rlap{$\square$}{\raisebox{2pt}{\large\hspace{1pt}\cmark}}%
\hspace{-2.5pt}}
\newcommand{\wontfix}{\rlap{$\square$}{\large\hspace{1pt}\xmark}}

%%% Varia %%% 
\usepackage{etoolbox}															% Extends TeX code
\usepackage{ifthen} 															% Produce if clauses
\usepackage{datetime2} 															% Get month and time
\makeatletter
\newcommand*{\monthyeardate}{%
  \DTMenglishmonthname{\@dtm@month}, \@dtm@year
}
\makeatother
\makeatletter
\newcommand*{\todaydate}{%
  \DTMenglishmonthname{\@dtm@month} \@dtm@day, \@dtm@year
}
\makeatother

%%% Format Headers/Footers and Chapters/Sections/Subsections/Subsubsections %%%%
%\usepackage[sf,sl,outermarks]{titlesec}
%\usepackage{titleps}
%\setcounter{secnumdepth}{2}														% Depth in Table of Contents
%\titlelabel{\thetitle.\quad}
\renewcommand*{\thepart}{\Roman{part}}
\renewcommand*{\thesection}{ \arabic{section}.}
\renewcommand*{\thesubsection}{\thesection \arabic{subsection}}
\renewcommand*{\thesubsubsection}{\roman{subsubsection}.}
%\numberwithin{equation}{section}
%\renewcommand*{\theequation}{\arabic{section}.\arabic{equation}}    

%%% Citing %%% 
%\usepackage[natbib=true]{biblatex} 											% Citing
\usepackage[authoryear]{natbib} 												% Citing
%\renewcommand{\bibsection}{}
%\usepackage{usebib}
%\usepackage{keyval}
%\setcitestyle{authoryear,aysep={},open={(},close={)}}
\usepackage{hyperref}															% Define links (color, shape, etc)

%\hypersetup{
%    pdffitwindow=false,									% Window fit to page when opened
%    pdfstartview={XYZ null null 1.00},					% Fits the zoom of the page to 100%
%    pdfnewwindow=true, 									% Links in new window
%    colorlinks=true,									% false: boxed links; true: colored links
%    linkcolor={red!50!black},							% Color of internal links (black is necessary for printing quality)
%    citecolor={blue!50!black},							% Color of links to bibliography
%    urlcolor=black,										% Color of external links
%%	hidelinks, 											% No colour nor borders in links
%	linkbordercolor={white},							% Color of links from hyperref; enables dropping of ugly borders in citations
%    pdfauthor = {Duarte Goncalves},
%    pdfkeywords = {1, 2, 3},
%    pdftitle = {\titlepresentation},
%    pdfsubject = {},
%    pdfpagemode = UseNone
%} 
\setlength{\bibsep}{0pt plus 0.3ex} 											% Determines space between entries in bibliography
\bibpunct{(}{)}{;}{a}{,}{,} 													% Citing Type; see natbib reference shee

%\addto\extrasenglish{															% Change \autoref automatic labels
%\renewcommand{\partautorefname}{chapter} 
%\renewcommand{\chapterautorefname}{section} 
%\renewcommand{\sectionautorefname}{section} 
%\renewcommand{\subsectionautorefname}{section} 
%\renewcommand{\figureautorefname}{fig.} 
%\renewcommand{\tableautorefname}{table} 
%\renewcommand{\equationautorefname}{eq.}
%}
%\renewcommand{\bibsection}{}					% References come at the end

\makeatletter
\DeclareRobustCommand\citepos													% Create possessives with \citepos[e.g.][p. Y]{Key}: e.g. Key's (Year, p. Y)
  {\begingroup\def\NAT@nmfmt##1{{\NAT@up##1's}}%
   \NAT@swafalse\let\NAT@ctype\z@\NAT@partrue
   \@ifstar{\NAT@fulltrue\NAT@citetp}{\NAT@fullfalse\NAT@citetp}}

\pretocmd{\NAT@citex}{%
  \let\NAT@hyper@\NAT@hyper@citex
  \def\NAT@postnote{#2}%
  \setcounter{NAT@total@cites}{0}%
  \setcounter{NAT@count@cites}{0}%
  \forcsvlist{\stepcounter{NAT@total@cites}\@gobble}{#3}}{}{}
\newcounter{NAT@total@cites}
\newcounter{NAT@count@cites}
\def\NAT@postnote{}

\def\NAT@hyper@citex#1{%														% Include postnote and \citet closing bracket in hyperlink
  \stepcounter{NAT@count@cites}%
  \hyper@natlinkstart{\@citeb\@extra@b@citeb}#1%
  \ifnumequal{\value{NAT@count@cites}}{\value{NAT@total@cites}}
    {\if*\NAT@postnote*\else\NAT@cmt\NAT@postnote\global\def\NAT@postnote{}\fi}{}%
  \ifNAT@swa\else\if\relax\NAT@date\relax
  \else\NAT@@close\global\let\NAT@nm\@empty\fi\fi								% Avoid compact citations
  \hyper@natlinkend}
\renewcommand\hyper@natlinkbreak[2]{#1}

\patchcmd{\NAT@citex}															% Avoid extraneous postnotes, closing brackets
  {\ifNAT@swa\else\if*#2*\else\NAT@cmt#2\fi
   \if\relax\NAT@date\relax\else\NAT@@close\fi\fi}{}{}{}
\patchcmd{\NAT@citex}
  {\if\relax\NAT@date\relax\NAT@def@citea\else\NAT@def@citea@close\fi}
  {\if\relax\NAT@date\relax\NAT@def@citea\else\NAT@def@citea@space\fi}{}{}
\patchcmd{\NAT@cite}{\if*#3*}{\if*\NAT@postnote*}{}{}

\makeatother


%% Other citation commands
%\cite{citationkey}: Author (Year)\\
%\citet[p. X]{citationkey}: Author (Year, p. X)\\
%\citep{citationkey}: (Author, Year)\\
%\citep[ch. Y]{citationkey}: (Author, Year, ch. Y)\\
%\citep[see][]{citationkey}: (see Author, Year)\\
%\citep[see][ch. Y]{citationkey}: (see Author, Year, ch. Y)\\
%\citepos[see][ch. Y]{citationkey}: see Author's (Year, ch. Y)\\
%\citealt{citationkey}: Author Year
%\defcitealias{KEY}{Alias}; \citetalias{KEY}: Alias\\
%\citetalias[Year, p.]{KEY}: Alias (Year, p.); \citepalias[Year, p.]{KEY}: (Alias, Year, p.)

\makeatletter
\setbeamertemplate{footline}
{
  \leavevmode%
  \hbox{%
  \begin{beamercolorbox}[wd=.25\paperwidth,ht=2.25ex,dp=1ex,center]{author in head/foot}%
    \usebeamerfont{author in head/foot}\footauthorpresentation
  \end{beamercolorbox}%
  \begin{beamercolorbox}[wd=.5\paperwidth,ht=2.25ex,dp=1ex,center]{title in head/foot}%
    \usebeamerfont{title in head/foot}\foottitlepresentation
  \end{beamercolorbox}%
  \begin{beamercolorbox}[wd=.25\paperwidth,ht=2.25ex,dp=1ex,right]{date in head/foot}%
    \usebeamerfont{date in head/foot}\insertshortdate{}\hspace*{2em}
    \insertframenumber{} %/ \inserttotalframenumber
    \hspace*{2ex} 
  \end{beamercolorbox}}%
  \vskip0pt%
}
\makeatother

\date{\datepresentation}
\title[]{\foottitlepresentation}
\author{\authorpresentation}
\institute{\institutepresentation}






\newtheoremstyle{named}{}{}{}{}{\bfseries}{.}{.5em}{\thmnote{#3}}
\theoremstyle{named}
\newtheorem*{namedenv}{namedenv}
%
%\makeatletter
%\declaretheoremstyle[
%	spaceabove=15pt,
%	spacebelow=15pt,
%%    headformat={\makebox[0pt][r]{\ }\NAME\, \NUMBER: \NOTE},
%    headformat=\NAME\, \NUMBER: \NOTE,
%    headpunct={},
%    notefont=\bfseries, 
%%  headfont=\normalfont,%\scshape,
%	notebraces={}{},
%	bodyfont=\normalfont,
%	postheadspace=\newline,
%	qed=\qedsymbol
%]{mythmstyle}
%\makeatother
%\declaretheorem[style=mythmstyle, title=Example]{examples}


